\documentclass[11pt]{article}
\usepackage[total={17cm, 24cm}]{geometry}                % See geometry.pdf to learn the layout options. There are lots.
\geometry{a4paper}                   % ... or a4paper or a5paper or ... 
%\geometry{landscape}                % Activate for for rotated page geometry
%\usepackage[parfill]{parskip}    % Activate to begin paragraphs with an empty line rather than an indent
\usepackage{graphicx}
\usepackage{amssymb}
\usepackage{epstopdf}
\usepackage[normalem]{ulem}
\usepackage{natbib}
\DeclareGraphicsRule{.tif}{png}{.png}{`convert #1 `dirname #1`/`basename #1 .tif`.png}

\usepackage{color}
\usepackage[
	bookmarks = true,
	bookmarksnumbered = false, 	% true means bookmarks in
							% navigation window are numbered
	bookmarksopen = false, 		% true means only level 1
							% are displayed.
	colorlinks = true,			% false for frames around links, true for color
	linkcolor = myred,
	citecolor = mygreen,
	urlcolor = myblue
	]{hyperref}
	
\definecolor{mygreen}{rgb}{0, 0.5, 0}	% less intense green
\definecolor{myblue}{rgb}{0, 0, 0.75}		% less intense blue
\definecolor{myred}{rgb}{0.75, 0, 0}		% less intense red
\definecolor{myrev1}{rgb}{1, 0, 0}		% less intense red; after approval, simply turn into black
\definecolor{temphidden}{rgb}{0, 0, 0}		% less intense red; to re-highlight items that finally need to be changed,
									% but should be black temporarilly
% \definecolor{myrev1}{rgb}{0, 0, 0}
\definecolor{mycorr1}{rgb}{0, 0, 1}		% less intense red; after approval, simply turn into black



% Define customised list environments.
\newenvironment{my_description}
{\begin{description}
  \setlength{\itemsep}{2pt}
  \setlength{\parskip}{0pt}
  \setlength{\parsep}{0pt}}
{\end{description}}

\newenvironment{my_enumerate}
{\begin{enumerate}
  \setlength{\itemsep}{2pt}
  \setlength{\parskip}{0pt}
  \setlength{\parsep}{0pt}}
{\end{enumerate}}

\newcommand{\ra}{$\rightarrow$\ }
\newcommand{\C}{\textbf{C:}\ }
\newcommand{\Q}{\textbf{Q:}\ }
\newcommand{\R}{\textbf{R:}\ }
\newcommand{\V}{\textbf{S:}\ }

\newcommand{\fst}{$F_{\mathrm{ST}}\ $}
\newcommand{\qst}{$Q_{\mathrm{ST}}\ $}


\title{Reply to de Villemereuil and Gaggiotti ``A new $F_{\mathrm{ST}}$-based method to uncover local adaptation using environmental variables''}
%\author{Simon Aeschbacher}
%\date{4 November 2014}                                           % Activate to display a given date or no date

\begin{document}
\maketitle
%\section{}
%\subsection{}

\section{Summary}
The authors of this study present an interesting solution to the problem of identifying candidate genes underlying local adaptation in the context of the so-called $F$ model. In this framework, the distribution of allele frequencies is modeled by a multinomial-Dirichlet likelihood (or, in the biallelic case, a beta-binomial likelihood), which is justified for the finite island model of population structure. The parameters of the distribution depend on a linearised form of $F_{\mathrm{ST}}$ that measures population differentiation at each locus and for each population. This linearised form is in turn modeled as a function of a locus- and a population-specific effect, as well as an interaction term. In principle, local adaptation would be detected if the interaction terms are significant. However, in realistic scenarios, there are too many combinations of loci and populations, so that the interaction terms cannot be estimated. The core idea of this paper is to model the interaction term as the product of the value of an observed environmental variable believed to be relevant for local adaptation, and a locus-specific quantity describing the sensitivity of a locus to its environment. Because the environmental variable is observed, the number of terms to be estimated scales only with the number of loci. This pays off if there are many populations. 

The authors extend a previous approach for inference under the $F$ model called BayeScan. BayeScan estimates only the locus- and population-specific coefficients. The extension presented here is called BayeScEnv and performs a comparison of three models. One is the neutral null model (population-specific effects only), the second one is a model of local adaptation without locus-specific effects, and the third one corresponds to the one implemented in BayeScan. For reasons not explained, the full model with all terms present, is not included. A procedure for model selection is presented, and a simulation study is conducted to compare BayeScEnv to BayeScan in terms of the false discovery rate, the false positive rate, and power. This is done for a number of models of population structure and demography that deviate from the island model. The performance of BayeScEnv depends strongly on the prior believes in the various models, as well as the type of population structure.

The approach is then applied to human data with the aim of identifying candidate genes for adaptation to high altitude. The environmental variables tested in turn are temperature, elevation above sea level, and precipitation. It is also applied to a dataset from North American salmon, where the environmental variables are temperature, precipitation, and river properties. In both cases, candidate genes are identified. In the case of the human data, an analysis for enrichment of gene ontology terms is performed. While some candidates are related to genes involved in processes that seem relevant to adaptation to high altitude, there is uncertainty about the majority of candidates identified.

A major weakness of this article is that the full model (with all terms preset) is not included. In addition, the simulation study is restricted to a few scenarios and does not cover major claims, such as that BayeScEnv is able to differentiate between local adaptation and background selection, variable mutation rates, or local adaptation with respect to unmeasured environmental variables.

\section{General issues}

\begin{my_enumerate}
	\item I think that the full model, $\log{(1/\theta_{ij})} = \alpha_i + \beta_j + g_i E_j$ (call it $\mathbf{M4}$), should be included in the set of models from which the `best' one is chosen. A justification for why $\mathbf{M4}$ has not been included is missing. I do not see why, in applications to real data, one should only allow either locus-by-environment interactions or locus-specific effects, but not both at the same time. It would be worthwhile to test whether, for instance, BayeScEnv chooses model $\mathbf{M2}$ and suggests that all the $\alpha_i$ are non-significanlty different from zero if the data were simulated under a scenario of local adaptation \emph{without} other locus-specific effects. The inverse should also be tested. I am curious to see if BayeScEnv can identify the models uniquely and what the false discovery rate would be.
		
	\item The authors claim that their approach is able to distinguish local adaptation from background selection and/or scenarios with non-uniform mutation rates across neutral sites (e.g.\ l.9, l.53--56). They also suggest that the method should pick up a signal of local adaptation due to an environmental factor other than the focal one (l.120--122). I would have liked to see a proof of these claims in terms of a simulation study. Currently, `only' alternative demographic models and scenarios of local adaptation are tested. I would be interested in a comparison of alternative modes of selection and variable mutation rates. That could well be restricted to the finite island model in this paper (see comments below).
	
	\item The $F$ model and its specific parametrisation used here have been introduced in a series of papers \citep{Takahata:1983bs, Balding:1996fp, Balding:2003eu, Beaumont:2004kx, Foll:2008uq}. Earlier articles \cite[e.g.][]{Takahata:1983bs, Balding:1996fp} emphasise that it is justified under the finite island model, but later work has applied it to a series of more complex models. The current paper does this, too. I acknowledge that these extended applications `work' in a statistical sense, producing rates of error that seem `acceptable' with respect to some (mostly) arbitrary threshold. Yet, I am missing a conceptual justification for such a liberal extension to other models of population structure and history. I am nervous that a deviation from the original assumption enhances the apparent problem of vast numbers of positive results produced by genome scans for selection. Whereas simulation studies can provide an idea of the false positive rate under a certain model, when these methods are applied to real data, there can only be limited confidence in the results. In the current paper, I find it very difficult to make sense of the large number of candidate genes identified with respect to precipitation and local adaptation.
		
	\item Related to the previous point, it seems difficult to draw general conclusions from comparisons of measures of performance of genome scans (e.g.\ in terms of false discovery or false positive rate, or power). This applies both to comparisons within a method among alternative models of population structure, as well as to comparisons across methods under a specific model of population structure. How does the apparent disregard of the original assumption of a finite island model interfere with performance? To what extent do various deviations from the island model affect comparisons within and among approaches? I realise that it might not be possible to fully address this and the previous point in the current article, but I miss an appropriate discussion of these issues.
	
	\item I am puzzled by the lack of agreement between this study and the one by \cite{Foll:2014fj} with respect to adaptation to altitude in humans. Do larger numbers of loci and an optimised sampling regime in \cite{Foll:2014fj} really account for the missing overlap in candidates? If so, how confident can we be that we learn about locad adaptation in humans from this study? To clarify this, BayeScEnv should be applied to the dataset used by \cite{Foll:2014fj}.%, but I want to leave this decision to the editor, as it means a substantial extension.

	\item I found it difficult to understand why, of all things, the largest difference in performance among the three BayeScEnv settings is between ($\pi = 0.1$, $p = 0.5$) and ($\pi = 0.1$, $p = 0$) [see comments below]. These two settings intuitively seem more similar to each other than to the third setting ($\pi = 0.5$, $p = 0.5$). Or am I missing something? In general, I find it rather problematic that the performance of the method depends relatively strongly on the prior believes about the models (choice of $\pi$ and $p$).
	
	\item Some parts of the paper are redundant (see below for details). I suggest an effort be made to remove these and shorten the main text. On the other hand, some sections of the Supplementary Information (SI) are only loosely related to  the main text and lack a motivational description at their beginning. The SI would profit from a revision aiming at a consistent text alignment (removing centered sections) and formatting that is consistent with the journal's requirements (e.g.\ numbering of figures).
	
	\item The writing seemed overly wordy to me at various places (see below for some suggestions). In addition, only redefine abbreviations (such as FPR or FDR) when really needed to remind the reader, e.g.\ at the beginning of a new section.
	
	\item I find that the abbreviations `BayeScan, `BayeScEnv' and `Bayenv' are misleadingly close, especially given that `Bayenv' is methodologically rather distinct. It also is not clear how BayeScEnv should be pronounced.
	
	\item I was a bit confused by the fact that BayeScEnv (and, for the matter, BayeScan) is classified as an `$F_{\mathrm{ST}}$-based method'. To me, this would imply that the approach tries to explain observed estimates of $F_{\mathrm{ST}}$ (i.e.\ the statistic $F_{\mathrm{ST}}$). However, here, $F^{ij}_{\mathrm{ST}}$ -- or rather its linearisation $\theta_{ij}$ -- is a parameter that is in turn modeled as a parametric function of $\alpha_i$, $\beta_j$, and $\gamma_{ij}$. I am fine with this construction called $F$ model, but I am worried about the confusion between $F_{\mathrm{ST}}$ as a statistic and a parameter, and would therefore avoid `$F_{\mathrm{ST}}$-based' method in this context.
	
	\item In genomic regions of low recombination rate or if selection is strong enough, linkage may spread out the signal of selection across several neutral markers. The approach presented here ignores such effects, as it assumes linkage equilibrium among loci. Although it is beyond the scope of this article to fully address this, I think it should be discussed at some point.
	
	\item The Discussion is too long and contains entire paragraphs that did not make sense or at least are not yet thought over very well. The writing also seemed preliminary in terms of language.
	
\end{my_enumerate}
	

\section{Specific comments}

\subsection{Abbreviations used}
\begin{my_description}
	\item[Q] Question % \Q
	\item[C] Comment % \C
	\item[S] Suggestion % \V
	\item[R] Re-formulation or change needed (usually followed by a suggestion) % \R
	\item[\ra] Suggested change/correction
\end{my_description}

\subsection{Title}
\C As mentioned above, I find `$F_{\mathrm{ST}}$-based approach' problematic in this context. This comment applies to later occurences of this phrase.

\subsection{Abstract}

\begin{my_description}
	\item[l.4] \R Move the comma from after to before `and'.
	\item[l.7] \V ```environmental differentiation"' \ra `environmental differences' or `differences in the environment'
	\item[l.7] \R `F model' \ra `$F$ model' (several occurrences, I will not repeat myself). \R Insert commas after `model' and 'but'
	\item[l.8] \R Insert comma after `approaches'; insert semicolon after `effects'; insert comma after `selection'.
	\item[l.12] \V `Human' \ra `human' (several occurences, I will not repeat myself). \V `Salmon' \ra `salmon' (several occurences, I will not repeat myself)
	
\end{my_description}

\subsection{Introduction}
\begin{my_description}
	
	\item[l.16] \V Delete `the new field of'.
	\item[l.17] \R `\dots selection in non-model species genomes\dots' \ra `\dots selection in genomes of non-model species\dots'
	\item[l.18--19] \R Insert comma after `adaptation'. \R `\dots populations experiencing different environemtnal conditions undergo an adaptive selective pressure specific to their local habitat.' \ra `\dots populations experience a selective pressure towards environmental conditions specific to their habitat.'
	\item[l.19--20] \V `As a result\dots conditions' \ra `As a result, populations evolve traits that provide an advantage in their local environment'
	\item[l.22] \V When citing \cite{Blanquart:2013uq}, please add `, and references therein' or similar. \R Insert comma before `but'.
	\item [l.21--23] \C My impression is that there are still few studies that convincingly make a link from selective agents to phenotype to underlying genotype. The advent of NGS on its own has not improved this. The hard part is in establishing the functional link via experiments. It would be good if you could be more explicit about what exactly made it possible `to make inference about the genetic architecture of [\dots] traits' underlying locally adaptivve phenotypes.
	\item[l.24] Delete `see'.
	\item[l.25] `\dots to selection, which are being frequently\dots' \ra `\dots to selection. These methods are now widely\dots'
	\item[l.25--26] \V When citing \cite{Faria:2014fk}, please add `, and references therein' (or similar).
	\item[l.30] \V `used by' \ra `behind'
	\item[l.32] \R `\dots as `outliers' loci.' \ra `\dots as outliers.'. \V I would quote `outlier(s)' only the first time it is used, i.e.\ in l.27.
	\item[l.35--36] \R `hypothesis' \ra `hypotheses'. \C To be fair, these methods have not been designed to identify the selective agents, but to statistically detect a signal of local adaptation.
	\item[l.39--40] \R `aims' \ra `aim'. \V `allele frequency patterns' \ra `patterns of allele frequencies'
	\item[l.40--42] \C In this sentence, `underlying' occurs three times. Please avoid this and shorten the sentence.
	\item[l.42--44] \V `trying to perform' \ra `performing'. \R Please say \emph{why} this is misleading and error-prone.
	\item[l.44] \V `This, existing methods\dots' \ra `Instead, existing methods\dots'
	\item[l.74] \V `\dots of this type of models\dots' \ra `\dots of these approaches\dots'
	\item[l.49--51] \R Please say \emph{why} this is only true if `levels of genetic drift are low'.
	\item[l.56] \Q What about genetic drift / population history as a potential confounding factor?
	\item[l.59] \V Insert comma after `Here'.
	\item[l.60] \V Insert comma after `local adaptation'.
	\item[l.61] \R `\dots allow for making inferences\dots' \ra `\dots allow for inference\dots'
	\item[l.66] \R Please check if the spaces after dots that denote abbreviations are not too large (here after e.g.). To prevent too large spaces from happening, type `e.g.\textbackslash XY' instead of `e.g. XY'. This occurs often and I will not repeat myself.
	\item[l.67] \R `The estimation of the two first terms benefit\dots' \ra `\dots The estimation of the first two terms benefits\dots'
	\item[l.68] \V `\dots populations, but this is not\dots' \ra `\dots populations; this is not\dots'
	\item[l.69] \V Move `therefore' to between `are' and `inferred'.
	\item[l.70] \V Delete `therefore,' to avoid repetition.
	\item[l.71] \V Remind the reader that `this regression term' is $\alpha_i$: `\dots this regression term.' \ra `\dots this regression term ($\alpha_i$).'.
	\item[l.73] \R You should also define $g_i$ at this point.
	\item[l.75] \C The acronym RJMCMC has not been defined before.
	\item[l.76] \R Insert comma after `follows'; `\dots our Bayesian approach, we then\dots' \ra `\dots our Bayesian approach. We then\dots'
	\item[l.77] \V Delete `then'.
	\item[l.78] \V `interest' \ra `implications'/`scope'/`limitations'?
		
\end{my_description}



\subsection{Statistical model}

\begin{my_description}
	
	\item[l.81] \R `genome scan approach' \ra `genome-scan approach'
	\item[l.89] \R `at the locus $i$' \ra `at locus $i$'
	\item[l.90] \R As defined here, $\theta$ does not measure differentiation, but similarity. The inverse of $\theta$ scales with differentiation. Please reformulate.
	\item[l.94] \V `e.g. AFLP-type of markers' \ra `e.g.\ AFSP markers'
	\item[l.95--96] \V `e.g. SNP markers' \ra `e.g.\ SNPs'
	\item[l.96--97] \V `Beta-binomial' \ra `beta-binomial'; `\dots the model is trying to estimate the $F_{\mathrm{IS}}$ as well as\dots' \ra `\dots the method estimates deviations from Hardy--Weinberg proportions ($F_\mathrm{IS}$) as well as\dots'. \R `\dots, see Foll and Gaggiotti (2008).' \ra `\dots (see Foll and Gaggiotti 2008).'
	\item[l.98] \C The phrase `$\theta_{ij}$-population structure' is not very meaningful in a section title. Please avoid it.
	\item[l.98--99] Here, I missed the presentation of the full model, $\log(1/\theta_{ij}) = \alpha_{i} + \beta_{j} + g_{i} E_{j}$ and, in the following, its incorporation into the approach (see general issue 1 above).
	\item[l.102] \R `assume' \ra `assumes'
	\item[l.107] \R `U-shape distribution' \ra `U-shaped distribution'
	\item[l.112] \V `\dots distance between the environmental value of population $j$ and\dots' \ra `\dots environmental difference between population $j$ and\dots'
	\item[l.113--115] \Q Why is the Manhattan distance in one dimension more robust to outliers than the Euclidean distance? \Q Why would this be an advantage (or of `major interest')? Is it because it makes detection of selection harder, and hence the approach more conservative in terms of indicating local adaptation? Please clarify.
	\item[l.114] `\dots the absolute value (i.e. Manhattan distance), whose major interest is that\dots' \ra `\dots the absolute value (i.e.\ Manhattan distance). Its advantage is that\dots'
	\item[l.116] \V Say how you standardise the environmental variables.
	\item[l.117--119] \V I did not like the formulation here. The sentence is too complicated, and, strictly speaking `where $g_i$ quantifies the effect of factor $E_j$ on the locus $i$' is wrong. If this were the case, then $g_i$ would have to have an index $j$, too. However, when you set $\gamma_{ij} = g_i E_j$, then $g_i$ modulates the effect of \emph{any} environment on locus $i$. This argument is somewhat hypothetical, as you only include one environmental variable at the time, but I am promoting it for conceptual coherence. Suggestion: `To incorporate local adaptation, we model the interaction term as $\gamma_{ij} = g_{i} E_{j}$, where $E_j$ quantifies the environment experienced by population $j$, and $g_i$ describes the sensitivity of locus $i$ to the environment.'
	\item[l.120] \V `due to our' \ra `with respect to the'
	\item[l.121--123] \C I find it not very satisfying that local adaptation with respect to an observed environmental factor is treated diffently from local adaptation with respect to an unobserved (unknown) factor. I understand that this is a consequence of the principal idea of the paper, i.e.\ of avoding the estimation of $i \times j$ terms $\gamma_{ij}$ by setting $\gamma_{ij} = g_i E_j$ instead. But it means that local adaptation with respect to an unknown factor cannot be identified with certainty.
This issue could be emphasised more, given that the same rationale was used to highlight the limitation of the preceding method (BayeScan) earlier on in the paper. BayeScEnv solves only part of the problem of uniquely identifying a signal of local adaptation. In applications where no obvious environmental variable responsible for local adaptation is known, BayeScEnv does not help.
Similarly, if a user misses out on the `true' (mechanistically responsible or statistically correlated) environmental variable(s) responsible for local adaptation, the model is forced to suck the existing signal into $\alpha_i$ and $\beta_j$ as well as possible. This may compromise the estimation of the `true' locus- and population-specific effects, and hence blur the overall inference. From that perspective, the two-step procedures (criticised here) in which a statistical signal of local adaptation is sought first, and a correlation with potential causal environmental variables established second, might be more justified.
I think these issues should be addressed/discussed.
	
	\item[General] \Q As a more general question, have you thought about fitting a model with multiple environmental variables (or, for the matter, principal components) at the same time? Could that be more informative than fitting models with one variable at the time?
	
\end{my_description}

\subsection{Material and Methods}

\begin{my_description}

	\item [l.138--139] \V Delete `absolute-value' (it is redundant with Manhattan distance). \C Again, I could not get my head around why the Manhattan distance should be less sensitive to outliers compared to e.g.\ the Euclidean distance.

	\item [l.140] \V Insert comma after `Also'. \R Please be more specific about these pilot studies.
	\item [l.141] \R `require' \ra `requires'
	\item [l.141--143] \C This sentence is without meaning. Please be more specific.
	\item [l.146--148] \C I think the general model (see above) should be included into the procedure.
	\item [l.150] \R `\dots inferences about them do not\dots' \ra `\dots inference about them does not\dots'
	\item [l.151--156] \C This part is redundant with previous paragraphs (also in the Introduction). I would prefer a compact description, as it is given here, much earlier on in the paper.
	\item [l.154--156] \V `Besides' \ra `By'. \R `\dots greatly reduced, making inference\dots' \ra `\dots greatly reduces. This makes inference\dots'; `' \ra `\dots than with\dots' \ra `\dots compared to\dots'. Insert `$\gamma_{ij}$' after `random coefficient'. \Q Is it correct to call this a `random coefficient'? The variables are random, the coefficients are fixed, but need to be estimated. I suspect the Bayesian perspective might have led to that slightly confusing formulation.
	
	\item [l.157--158] \V `3' \ra `three' (The same applies to all numbers smaller than 13 in a textual context). \R `\dots preceded by pilot runs aimed at\dots' \ra `\dots preceeded by pilot runs. These are aimed at\dots'.
	\item [l.159] \V Insert comma after `rates'. \Q Should `approximating the posterior distribution of parameters' not be the job of the RJMCMC, rather than the pilot runs? I would appreciate if the pilot runs were described in some more detail, perhaps just in the SI.
	\item [l.164] \V `3' \ra `three'; `2' \ra `two'
	\item [l.165] \V `\dots for each model:' \ra `\dots for each model as'; add commas after the first and second line in Eq.\ (5) and a full stop after the third.
	\item [l.167--168] \R `\dots conservative results.' \ra `\dots conservative results with respect to detecting local adaptation (choosing model $\mathbf{M2}$).'. \C I was nervous about such strong a prior belief in model $\mathbf{M2}$ (local adaptation). I was confused about the fact that you try to be \emph{less} conservative here by choosing a low $p$, but you have chosen the Manhatten distance instead of the Euclidean distance above, arguably to be \emph{more} conservative.
	\item [l.169--171] \C It would be nice to see a plot of the priors in the SI. In particular, I wonder if the approximation by a Normal distribution is justified. Is there no skew? Also, a mean prior $F^{ij}_{\mathrm{ST}}$ of 0.27 seems pretty high to me. Did I miss something?
	\item [l.172--174] This is redundant to previous text.
	\item [l.177] \R Insert space before `for'; delete colon after `as'; add full stop at the end of Eq.\ (6)
	\item [l.178] \C It would be helpful to have a motivational sentence about $q$-values here. \R Delete repeated `from'.
	\item [l.183--184] \V `In order\dots decide, our code\dots' \ra `Our code\dots'
	\item [l.191] \V `significance $\alpha$ threshold' \ra `significance threshold $\alpha$'
	\item [l.191] \Q Why are you using the less conservative measure of performance ($q$-values instead of PEP)? How sensitive are your conclusions to the choice of the measure of performance?
	\item [l.198] \R `1' \ra `one'; `\dots were selected.' \ra `\dots were under selection.'
	\item [l.199--200] \V `\dots Island Model\dots' \ra `\dots island model\dots'; `\dots one-dimension Stepping-Stone\dots' \ra `\dots one-dimensional stepping-stone\dots'
	\item [l.201] \R Delete the sentence `For all scenarios,\dots'. It is redundant with l.197. `10' \ra `ten'
	\item [l.202] \V `The selected loci\dots' \ra `The loci under selection\dots'
	\item [l.206] `Beta-binomial' \ra `beta-binomial'
	\item [l.207] `\dots each markers were\dots' \ra `\dots each marker was\dots'
	\item [l.207] \V Use `IM' and `SS' only, as the abbreviations have just been defined.
	\item [l.213] \R Please specify what should be `particularly difficult for' your method.
	\item [l.216] \C It was not clear to me what exactly you mean by `populaton structure gradient'.
	\item [l.218] \C At least to me, `logit-like transformation' does not seem to be an established term. Is the transofrmation in Eq.\ (8) an \emph{ad hoc} approach? If so, please state why it is used.
	\item [l.221] \R Please add a comma at the end of the equation, and then `where $n_{11}$ and $n_{00}$ are the number of loci at which the individual is homozygous for the advantageous and disadvantageous allele, respectively'.
	\item [l.222] \R `\dots scripts are availalbe online': Please give the URL
	\item [l.223] \R Please say where the code can be found.
	\item [l.225] \V `\dots our simulations yielded\dots' \ra `\dots we observed\dots'; add `in our simulations' after `disequilibrium'
	\item [l.236] \R `consists into' \ra `consists of'
	\item [l.237] \Q What was the effect of removing low-frequency alleles?
	\item [l.238] \V Insert `the following' after `obtained'
	\item [l.240---241] \V `3' \ra `three'. This sentence seemed redundant to previous text.
	\item [l.243] \R `test' \ra `tests'; `was' \ra `were'
	\item [l.252] `precipitations' \ra `precipitation'
	\item [l.257] Delete `of' before `10'.
	
\end{my_description}

\subsection{Results}
\begin{my_description}
	\item[l.261] \V `a FDR' \ra `an FDR'
	\item[l.262] \C It is not fully clear what you mean by `both methods'; BayeScEnv and BayeScan, or PEP and $q$-values?
	\item[l.263] \V Add comma after `calibrated'.
	\item[l.265] \R Insert space after `i.e.'.
	\item[l.268] \V Insert comma after `BayeScan'.
	\item[l.269--270] \C It seems that the model priors ($p$ and $\pi$) have a considerable impact on the FDR (and FPR). I do not think that this is particularly encouraging in terms of application to real data.
	\item[l.283--285] \Q Would a full model $\mathbf{M4}$ (see above) pick up more loci at all, but be able to differentiate between locus-specific and environment-specific effects? See also general issue 1 above.
	\item[l.302--304] \Q Importantly, this shows the weaknes of BayeScan, but can BayeScEnv distinguish a full model ($\mathbf{M4}$) from $\mathbf{M2}$ and $\mathbf{M3}$?
	\item[l.304] Delete comma after `pattern'.
	\item[l.311--312] \C I was confused by these numbers of singificant SNPs. I would expect that if you increase $p$ from 0 to 1 in BayeScEnv (giving increasing prior weight to the BayeScan scenario $\mathbf{M3}$, the number of significant SNPs should increase and ultimately be the same as with BayeScan. However, you observe a decrease in the number of significant SNPs when you increase $p$ from $0$ to $0.5$. Please clarify.
	
	\item [General] \C I would find it helpful to have a summary in the IS of estimates of $\alpha_i$, $\beta_j$, and $g_i$, at least for the simulation study, but ideally also for the two applications to real datasets.
		
\end{my_description}


\subsection{Discussion}
\begin{my_description}
	\item[l.324-325] \C As mentioned before (general issue 2), I would have liked to see a demonstration of how BayeScEnv detects local adaptation due to an unknown environmental variable, differences in mutation rates, or background selection.
	\item[l.327--328] \R Delete `\dots, as well as our data analyses\dots' (You could only measure the FDR in the simulation study!).
	\item[l.334] \V `non negligible' \ra `non-negligible'
	\item[l.337] \V `non neutral' \ra `non-neutral'
	\item[l.337--339] \C I did not fully understand what you mean. Please explain this further.
	\item[l.340--346] \C I find it problematic (but not surprising) that the performance of BayeScEnv depends so strongly on the demographic model. It means that one cannot be confident in results obtained with BayeScEnv unless one knows the demography / kind of population structure and has chosen the appropriate prior on $p$ and $\pi$.
	\item[l.344] \R `stepping stone model' \ra `stepping-stone model'
	\item[l.351] \R Delete comma after `BayeScan3'.
	\item[l.352] \V Delete `processes'
	\item[l.353--355] \C This is an overstatement in my view (see general issues 1, 2, and 4 above).
	\item[l.356] \V `How to define the `environmental differentiation'?' \ra `How to measure environmental difference?'
	\item[l.357--358] \V `\dots, we compute an ``environmental differentiation'' as a measure of distance (here the Manhattan, absolute value distance) to a reference\dots' \ra `\dots, we chose to quantify environmental differences in terms of the Manhattan distance from a reference\dots'
	\item[l.359] \R `kind' \ra `kinds'
	\item[l.361] \V Insert comma after `For example'; `\dots in the case of analysis focused on elevation\dots' \ra `\dots in our analysis of the effect of elevation in humans\dots'
	\item[l.362--364] \C I did not understand why the reference matters as long as you mean-center the values. I would argue that the scale matters, and I understand that a non-linear effect might be an issue when using the mean altitude as a reference. Please clarify.
	\item[l.368] \V `Principal Component Analysis' \ra `principal component analysis'
	\item[l.373] \Q What do you mean by `reduced'? Are you referring to a reduction in dimensionality via PCA? Please clarify.
	\item[l.374--375] \V Insert comma after `Also'. \R `for the ease' \ra `based on the assumption'
	\item[l.386--389] \C This did not make sense at all. If demography were to cause a deviation from independence between populations, then all loci should be affected in the neutral case. Therefore, it is not the locus-specific effect ($\alpha_i$), but the population-specific effect ($\beta_j$) that would give information about a deviation from the island model. Please rectify.
	\item[l.393--394] \C Looking at Eqs.\ (8) and (9), there is a specific functional form linking the environmental values to allele frequencies. Actually, you say in the next sentence that the relationship is exponential. Please rectify.
	\item[l.399--400] \R Insert `a' after `Such'; insert `, for instance,' after `arise'; `target of the selection' \ra `target of selection'; `\dots or regulatory genes for the response to stress.' \ra `\dots or genes regulating stress response.'
	\item[l.403--407] \C Such a pattern is only difficult to relate to local adaptation if one can be sure that the environmental variable that was measured is also the one responsible for local adaptation. However, the environmental variable relevant to local adaptation might simply not have been measured.
	\item[l.407] \R `Yet such signal should\dots' \ra `Yet, such a signal should\dots'
	\item[l.408] `\dots would explain such pattern would\dots' \ra `\dots could explain such a pattern would\dots'
	\item[l.404--414] \C This paragraph seemed somewhat tangential, and certainly very speculative, given that no proof of example is given.
	\item[l.415] \V `genetic-environment association methods' \ra `methods to study gene-environment associations'
	\item[l.420--422] \R `In accordance with this expectation, its application to the Human dataset identified several genomic regions that gene ontology enrichment identified as relevant to\dots' \ra `When applied to the human dataset, BayeScEnv identified several genomic regions that are enriched for gene ontology terms relevant to\dots'
	\item[l.423] \R `Human' \ra `human'
	\item[l.434] \R Start a new sentence before `other genomic significant regions\dots'
	\item[l.435] \R `non coding' \ra `non-coding'
	\item[l.437--441] \C I could not make much sense of this paragraph, other than being concerned about false positives. The remark in the parentheses is too long.
	\item[l.439] \R `Human' \ra `human'
	\item[l.454] \V Insert commas after `Because' and `adaptation'
	\item[l.456] \V `The Atlantic salmon genome is poorly annotated so we could not\dots' \ra `As the Atlantic salmon genome is poorly annotated, we could not\dots'
	\item[l.459--461] \C I would be interested in the number of candidates suggested by the full model ($\mathbf{M4}$, see above).
	
\end{my_description}

\section{Conclusion}
\begin{my_description}
	\item[l.463] \V Insert comma after `BayeScEnv'. \C As mentioned earlier, I have a slight reservation against these approches being referred to as $F_{\mathrm{ST}}$ (see comments above).
	\item[l.475] `F-model' \ra `$F$ model'
\end{my_description}
	
\section{References}
\begin{my_description}
	\item[General] \C Some entries contain fields such as `WOS:000322886400008'. Please check if their inclusion is intended.
	\item[l.504] \R `lewontin and krakauer test' \ra `Lewontin and Krakauer test'
	\item[l.506] \R `alpine' \ra `Alpine'(?)
	\item[l.507--508] \R `atlantic' \ra `Atlantic'(?). \Q Is this type of reference (`data from') accepted by the journal?
	\item[l.509--510] \R `atlantic salmon (salmo salar)' \ra `Atlantic salmon (\emph{Salmo salar})'
	\item[l.513] \R Ditto
	\item[l.516] \R `arabidopsis thaliana' \ra `\emph{Arabidopsis thaliana}'
	\item[l.517] \R `Markov chain monte carlo' \ra `Markov chain Monte Carlo'
	\item[l.521] \R Please check the title
	\item[l.523] \R Delete space before `--'.
	\item[l.529] \R `bayesian' \ra `Bayesian'; please fix the page range
	\item[l.530] \R Check if $QST$ and $FST$ should really be typed like this (all uppercase).
	\item[l.536] \R `bayesian' \ra `Bayesian'
	\item[l.538] \R Please check the issue number. I think it should be 4 instead of 3.
	\item[l.544] \V `f-model' \ra `$F$-model'
	\item[l.546] \R `markov chain monte carlo' \ra `Markov chain Monte Carlo'; `bayesian' \ra `Bayesian'
	\item[l.549] \R `. a fast\dots' \ra `. A fast\dots'
	\item[l.569] \R `bayesian' \ra `Bayesian'
	\item[l.583] Ditto; \V `q-value' \ra `$q$-value' (?)
	
\end{my_description}

%\section{Data Accessibility}
%Please improve the formulation. The current description reads as if the contribution is insignificant.

\section{Figure captions}
\begin{my_description}
	\item[Fig.\ 2] \V `False Discovery Rate' \ra `False discovery rate'; \R Delete space after `red'.
	\item[Figs.\ 2--4]\R `significance $\alpha$ threshold' \ra `significance threshold $\alpha$'; `Island model' \ra `island model'; `Stepping-Stone mdoel' \ra `stepping-stone model'; `Hierarchically Structured model' \ra `hierarchically structured model'; Insert `$\mathbf{(M3)}$', `$\mathbf{(M2)}$', and `$\mathbf{(M1)}$' after `locus-specific model', `environmental model', and `neutral one', respectively.
	\item[Fig.\ 4] \V `False Positive Rate' \ra `false positive rate'
	\item[Fig.\ 5] \V `Human dataset' \ra `human dataset'. \Q Would there also be stripes visible in panels A)--C) if more SNPs were significant, or are the stripes specific to panel D) (and BayeScan)? What do the horizontal lines mean?
	
\end{my_description}

\section{Figures}
\begin{my_description}
	\item[Figs.\ 2--4] \C If these figures are to be printed in grey scales, please increase the contrast between them.
	\item[Fig.\ 2] \C I found it difficult to understand why $\mathrm{FDR}_{(\pi = 0.1, p = 0)} > \mathrm{FDR}_{(\pi = 0.5, p = 0.5)} > \mathrm{FDR}_{(\pi = 0.1, p = 0.5)}$. The settings $(\pi = 0.1, p = 0)$ and $(\pi = 0.1, p = 0.5)$ seem more similar to each other than to $(\pi = 0.5, p = 0.5)$. \R `Bayescan' \ra `BayeScan' (legend)
	\item[Fig.\ 3] \Q Given that `BayeScEnv' with ($\pi = 0.1$, $p = 0.5$) is the scenario closest to `BayeScan', I found it surprising that these two differ most in terms of power. Do you have an explanation?
	\item[Fig.\ 5] \R Please annotate the $x$ axis.
\end{my_description}

\section{Tables}
\begin{my_description}
	\item[Table 1] \V `Results from the BayeScan and BayeScEnv on the Human dataset.' \ra `Results from BayeScan and BayeScEnv when applied to the human dataset.'; `FDR significance threshold was set to 5\%' \ra `An FDR (in terms of $q$-values) of 5\% was tolerated.'
\end{my_description}

\section{Supplementary Information}
\begin{my_description}

	\item[General] \R Please check the journal's requirements for the format of figure captions and the numbering of figures (Are Roman numbers allowed?).\\
	\R Please check if the spaces after dots that denote abbreviations are not too large. For instance, it seems that in `Fig.\ XY' you often use too large a space (LaTeX by default makes a space larger after a dot, as it assumes you start a new sentence). To prevent this from happening, type `Fig.\textbackslash XY' instead of `Fig. XY'.\\
	\cite{Guenther:2013ve} spell the name of their approach Bayenv, not BayEnv.
	
	\item[Section 1] \V `Recall that we have 3 models\dots posterior probabilities:' \ra `Recall the three models of which we want to infer posterior probabilities:'.\\
	\V When describing the models, add the left-hand side of the equations, e.g.\ write $\log{(1/\theta_{ij})} = \beta_j$ instead of just $\beta_j$.\\
	\V `\dots prior probability of model 2\dots' \ra `\dots prior probability of model $\mathbf{M2}$ \dots' and analogously for the rest of the text to be consistent in using abbreviations, e.g.\ `\dots going from 1 or 2\dots' \ra `\dots going from $\mathbf{M1}$ to $\mathbf{M2}$ \dots'.\\
	\C The sentence `We assume that the probability of going\dots transition matrix :' is too complicated; pease simplify. Also, delete the space in front of the colon.\\
	\R This whole section is not consistent with l.\ 146--148 and Eq.\ (5) of the main text (models $\mathbf{M2}$ and $\mathbf{M3}$ exchanged). Please rectify. Please also mention what the entries of the matrix in Eq.\ (1) of the SI mean. For instance, it is not immediately clear if the second entry in the first row is the probability of going from $\mathbf{M1}$ to $\mathbf{M2}$ or \emph{vice versa}.\\
	\C It was not clear to me how you get from Eq.\ (2) in the SI to Eq.\ (3) in the SI. Should each line in the latter correspond to the sum of the entries in each row of the former?\\
	\Q What do you mean by `$\pi \rightarrow 0$'? I suspect you mean that you ignore terms of order $\pi^2$ and higher, i.e.\ you assume that $\pi \ll 1$ (the current notation means you take the limit, which does not seem to be what you do).\\
	\R `\dots of each models \dots' \ra '\dots of each model'.
	
	\item[Section 2] \R Replace semicolon in front of `Gelman \emph{et al.} (2004)' by `and'.\\
	\V `\dots jump between model $l$ and $k$ \dots' \ra `jump from model $l$ to $k$ \dots'.\\
	\V Insert comma after $\mathrm{min}(r,1)$.\\
	\Q Do you mean `proposal kernel' instead of `kernel proposal'?\\
	\C Strictly speaking, a likelihood always refers to (a) parameter(s). Hence, I would avoid `the likelihood of the data $Y$ given the parameters\dots'. Correct statements are either `the probability of the data $Y$ given the parameters' or `the likelihood of the parameters (and the model) given the data'.\\
	\V Insert comma before the `and' in `\dots the model $M_{\bullet}$ and $J$ is the \dots'.\\
	\R Please use consistent notation for the model priors in Eqs.\ (3) and (4) of the SI, i.e.\ either use lowercase or uppercase $p$.\\
	\Q In these pilot runs, do you not also need to propose values for $u_{\bullet}$? How do you do this? Moreover, is it generally justified to approximate the pilot posterior by a Normal distribution?
	
	\item[Section 3] \V `Hierarchically Structured' \ra `Hierarchically structured' (section title).\\
	\R `Below (Fig.\ I) is the schematic representation\dots' \ra `Figure II gives a schematic representation\dots'.\\
	\C As you show `only some illustrative migration combinations' it is not clear what scheme exactly you simulated. Please clarify.\\
	
	\item[Section 4] \R Delete space in front of colon.\\
	\R `Stepping Stone model' \ra `Stepping-stone model'\\
	\R `Hierarchically Structured model' \ra `Hierarchically structured model'
	\R `\dots only for the Island model\dots' \ra `\dots only for the island model\dots'
	
	\item[Section 5] \C This seems to be the information about the environmental variable(s) that I was missing above in the main text where the simulation study is described. If so, please add a reference to the main text.\\
	
	\item[Section 6] \R Title and text: `False Positive Rate' \ra `false positive rate' and `False Discovery Rate' \ra `false discovery rate'.\\
	\R `Bayescan' \ra `BayeScan' (several occurrences).\\
	\R `\dots less false positives.' \ra `\dots fewer false positives.'\\
	\C Again, I find it difficult to develop an intuition for why just the parameterizations $\pi = 0.1, p = 0.5$ and $\pi = 0.1, p = 0$ show the biggest difference in the FPR (similar to the pattern for the FDR).
	\R In the caption of Fig.\ III: `significance $\alpha$ threshold' \ra `significance threshold $\alpha$'; `Island model' \ra `island model'; `Stepping-Stone model' \ra `stepping-stone model'; `Hierarchically Structured model' \ra `hierarchically structured model'; Delete space between `red' and ')'.\\
	
	\item[Section 7] \V `Below are the list \dots criteria.' \ra `Below is a list of the genes that fulfill the following two criteria.'\\
	\R `\dots there is at least a significalt SNPs in their neighbourhood, indicating them as\dots' \ra `\dots there is at least one SNP in their neighbourhood for which a significant gene-by-environment interaction (non-zero value of $g$) was inferred, which identifies them as \dots'; `\dots were found as significantly \dots' \ra `\dots were found to be significantly enriched\dots'.\\
	\R Please revise the paragraph starting with `Note that the majority\dots'. The first sentence is not clear, the second one must be reformulated to be precise, and the third one is missing an `of'. Please also provide the link or reference to the data files mentioned. I could not access these files for review. Last, `Bayescan' \ra `BayeScan'.
	
	\item[Section 8] \C The first sentence implies that the other methods could not have been used in the context of the scenarios use in the main text. Is this true? If so, please explain why the other methods cannot be used to study the former cases.\\
	\R Delete `study' after `de Villemereuil \emph{et al.} (2014)'.\\
	\C The description of the four additional scenarios is too short. I understand that you do not want to repeat de Villemereuil \emph{et al.} (2014), but you should go into some more detail. \C The statements `clinal environment(,) following population structure' and `a random environment strongly correlated with population structure' did not make sense to me.\\
	\R `Isolation with Migration Model' \ra `Isolation-with-migration model'; `Stepping-Stone model' \ra `Stepping-stone model'
	\V Please remind the reader of why these scenarios are `very difficult for all methods'.
	\V `\dots that False Discovery Rate (FDR) depends\dots' \ra `\dots that the false discovery rate (FDR) depends\dots'; `\dots the False Positive Rate (FPR) and\dots' \ra `\dots the false positive rate (FPR) and\dots'.\\
	
	\R `Fig. IV--VI' \ra `Figs.\ IV--VI'; `\dots in very difficult conditions\dots' \ra `\dots under very difficult conditions\dots'; \V `honourably' smells as if something is wiped under the carpet. Indeed, I would say BayeScEnv's overall performance is intermediate.\\
	\R Please provide an explanation for why BayeScEnv has much more power with $p=0$ than with $p=0.5$.\\
	\C I find it very hard to draw general conclusions about how well BayeScEnv performs. The results seem to depend a lot on the specific model and even the priors ($p$, $\pi$).\\
	\R 'False Discovery Rate' \ra `false discovery rate' (several occurrences); 'False Positive Rate' \ra `false positive rate' (several occurrences)\\
	\R `F-model family' \ra `$F$-model family'.\\
	Captions of Figs.\ IV and V: `False Discovery Rate' \ra `False discovery rate'; `significance $\alpha$ threshold' \ra `significance threshold $\alpha$'; `False Positive Rate' \ra `False positive rate'.
	
	\item[Section 9]
	\C This section would profit from a few motivational sentences. As it is, it seems detached from the rest of the paper, except that it illustrates the scenarios already briefly discussed in the main text.\\
	\C `Difficult scenario' is not a very informative term. Could you use a more descriptive one?\\
	\R `In this scenario populations with\dots' \ra `In this scenario, populations with\dots'.\\
	\Q False positives in what sense? Local adaptation or population-specific effects? I suspect the latter, but am not sure.
	
\end{my_description}

\bibliographystyle{/Users/Simon/Documents/Literature/Reviewing/FountainEtAl2013MolEcol/genetics}
\bibliography{/Users/Simon/Documents/Literature/BibDesk/Jshort_dot,/Users/Simon/Documents/Literature/BibDesk/central}
\end{document}  